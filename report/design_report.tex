\documentclass[a4paper,11pt,column]{article}
\usepackage[latin1]{inputenc}
\usepackage[english]{babel}
\usepackage{amsmath}
\usepackage{amsfonts}
\usepackage{amssymb}
\usepackage{wasysym}
\usepackage{cuted}
\usepackage{pdfpages}

\usepackage{titling}
\usepackage{siunitx}
\usepackage[style=ieee,backend=bibtex]{biblatex}
\addbibresource{designbib.bib}
\usepackage[font={small}]{caption}
\usepackage{subfig}
\usepackage{nomencl}
\makenomenclature





\usepackage{graphicx}
\usepackage{color}

\usepackage{booktabs}
\usepackage{threeparttable}
\usepackage{fancyhdr}
\usepackage{float} %floats are used to contain things that must be placed inside a single page

\usepackage{varioref}
\usepackage{textcomp}
\usepackage{xspace}
\usepackage[activate={true,nocompatibility},final,tracking=true,kerning=true,spacing=nonfrench,factor=1100,stretch=10,shrink=10]{microtype}




%---------------------------------------------------------------------------------------------------------------------------------------------------------------------------------------------------------
% Document Information
%---------------------------------------------------------------------------------------------------------------------------------------------------------------------------------------------------------
\author{Group 25}
\title{Design Report}
\date{\today}

% Path to Images.
\graphicspath{Images}

% Header and footer.
\pagestyle{fancy}
\fancyhf{}
\lhead{\thetitle}
\rhead{\theauthor}
\cfoot{\thepage}
\renewcommand{\headrulewidth}{0pt}
\renewcommand{\footrulewidth}{0pt}

\begin{document}


%% Title page.
\begin{titlepage}
    \centering
    \vspace*{\fill}
    \includegraphics[width=\textwidth]{Images/Durham.jpg}
    \vspace*{\fill}
    \LARGE\thetitle\\
    \large\theauthor\\
    \large P.H. Gaskell, K. Blundy\\
    \large L2 Design\\
    \large\thedate\\
    \vspace*{\fill}
\end{titlepage}

%---------------------------------------------------------------------------------------------------------------------------------------------------------------------------------------------------------
% Main matter.
%---------------------------------------------------------------------------------------------------------------------------------------------------------------------------------------------------------
%---------------------------------------------------------------------------------------------------------------------------------------------------------------------------------------------------------
% SUMMARY - ELOISE
%---------------------------------------------------------------------------------------------------------------------------------------------------------------------------------------------------------
\onecolumn 
\section{Executive Summary} 
Microphonic effects exist in hi-fi systems wherein mechanical vibrations are 
transformed into unwanted electrical signals, termed noise, that consequently 
reduce sound quality. Isofonics have designed footers of aluminium and steel 
upon which such systems are to be supported that aim to isolate and attenuate 
the signals responsible for this aforementioned noise. Three footers are 
required per system, three being the minimum number of points of contact 
necessary to support a rigid mass, and are to be sold as a device comprising a 
contained magnetic damping mechanism with two optional additional pieces: a 
removable base to use the device as a stand alone mount as opposed to its 
mounting within existing equipment and a removable spike offering a minimal 
point of contact. The main device, base and spike are to cost \textsterling 
blah, \textsterling blah and \textsterling blah respectively, placing Isofonics 
at the higher end of the existing market due to its premium quality components 
embodying a design that is both innovative and effective. The device may be 
altered manually by the user to support a range of different masses, a feature 
of customisability unique to Isofonics. This process may also be informed by a 
user friendly phone application. Additionally, within this design lies 
potential for product families of differing size, catering to an even wider 
audience, and applications within the field of optics, more specifically in the 
optimisation of microscopes.

% Acronyms
\nomenclature[A0]{DFM}{Design for Manufacture}
\nomenclature[A1]{DFA}{Design for Assembly}
\nomenclature[A2]{CNC}{Computer Numerical Control}
\nomenclature[A3]{URS}{User Requirement Specification}

% Variables
\nomenclature{test}{This is a test}

\printnomenclature

\tableofcontents

%---------------------------------------------------------------------------------------------------------------------------------------------------------------------------------------------------------
% INTRO - ELOISE
%---------------------------------------------------------------------------------------------------------------------------------------------------------------------------------------------------------

\section{Introduction}
Microphonics describes the phenomenon wherein internal components within an 
electronic device transform mechanical vibrations into undesired electrical 
signals\cite{microphonics}. In the context of hi-fi systems, and when these 
vibrations are within the frequency range audible to the human 
ear---20~Hz--20~kHz---they equate to noise that reduces audio quality and 
therefore 
threatens 
the user\textquotesingle s listening experience. This report details the design 
of vibration isolation and attenuation mounts that minimise this interference 
as well as that originating from external sources. 

The following function analysis tree defines the user requirement specification 
for the product:

\begin{figure}[h]
   \centering
   \includegraphics[width=\textwidth]{Images/URSTree}
   \caption{User Requirement Specification Diagram}
   \label{}
\end{figure}

With such a niche product consisting of high tolerance machined parts and 
premium materials comes a high cost, indicating a corresponding market; middle 
aged/mature clientele with strong technical understanding of the hobby and its 
principles or affluent younger hobbyists. The current market offers a range of 
mounts at a range of prices within which our product shall lie at the higher 
end, namely between \textcolor{red}{\textsterling 1500--\textsterling 2000} 
per mount. 

Market leaders such as Stillpoints\textsuperscript{\textregistered}, 
Nordost\textsuperscript{\textregistered} and 
Isoclear\textsuperscript{\textregistered} offer sleek solutions yet lack the 
tailorability offered by our design, allowing customers to pre-load their 
mounts manually for individual amps, potentially with interactivity provided by 
an instructive technical app. After conducting market research via surveys, it 
was found that potential customers were very much interested in the ability to 
manually adjust their mount as appropriate and that this offered an invaluable, 
unique selling point.

This report covers the initial design concepts proposed by Group 25, the 
development of a chosen design, its design for manufacture and sustainability 
and its commercial considerations.

%---------------------------------------------------------------------------------------------------------------------------------------------------------------------------------------------------------
% DESIGN CONCEPTS
%---------------------------------------------------------------------------------------------------------------------------------------------------------------------------------------------------------
\section{Design Concepts}

During initial research it was noticed the most successful products on the 
market met three primary criteria:

\begin{itemize}
    \item They allowed the mounted equipment to move with four degrees of 
    freedom: three in translation and one in rotation.
    \item They used single degree of freedom systems for damping. Simpler 
    damping mechanisms were more effective at consistently attenuating 
    unwanted frequency components.
    \item Vibrations were efficiently transmitted to the damping system from 
    the equipment. This was typically achieved using hard materials and limited 
    contact between the footer and equipment.
\end{itemize}

In summary, any concepts generated had to first \emph{isolate} the equipment 
then \emph{attenuate} vibrations by allowing the equipment to move freely and 
then damp vibrations in controlled manner.

In order to simplify the process of concept generation, the formulation of 
isolation and attenuation mechanisms was separated. Isolation mechanisms were 
designed to translate the four degrees of freedom in the movement of the 
equipment into the single degree of freedom of the damping mechanism. Those 
single degree of freedom damping mechanisms were designed to be easily applied 
to multiple isolation concepts.

\subsection{Damping Mechanisms} \label{sec:damping_mechanisms}

Three methods of damping were considered---not including the friction in 
bearings, which cannot be completely eliminated. These were viscoelastic, 
magnetic and fluid damping.

Viscoelastic materials, such as rubber or Sorbothane\textregistered, heat up 
when subjected to changes in mechanical stress \cite{meyers2009mechanical}. 
Hence, the amplitude of vibrations are reduced when the driving oscillations 
are transmitted through viscoelastic materials. Typically when these materials 
are used in footers, the vibrations are not limited to a single degree of 
freedom, so the effectiveness of attenuation depends on the vibration 
direction. This makes it difficult to model and specify a single system natural 
frequency or damping factor.

Figure~\vref{fig:viscoelastic} details how viscoelastic materials can be used 
in a system where vibrations have been translated to a single direction. A 
linear bearing compresses a viscoelastic sleeve on the same shaft.

\begin{figure}[h] 
    \centering
    \textbf{VISCOELASTIC DAMPING CONCEPT}
    \caption{D1---Viscoelastic damping mechanism.}
    \label{fig:viscoelastic}
\end{figure}

Magnets can also be used to damp vibrations when constrained to a single 
direction. As a permanent magnet travels through a coil, an EMF is induced due 
to Faraday's law, which drives a current when the circuit is complete. The 
current flowing generates a magnetic field opposing the permanent magnet, 
according to Lenz's law. A possible assembly for this system is detailed in 
Figure~\vref{fig:magnetic}. The strength of magnets were to be tuned to the 
desired dynamic properties of the system.

\begin{figure}[h] 
    \centering
    \textbf{MAGNETIC DAMPING CONCEPT}
    \caption{D2---Magnetic damping mechanism.}
    \label{fig:magnetic}
\end{figure}

Finally, a simple fluid damping system was considered, whereupon a bearing 
travels through some viscous fluid. Skin friction provides the damping force in 
Figure~\vref{fig:fluidic}. The size of the bearing could be tuned to meet the 
required constraints for damping factor and natural frequency. However, this 
concept was difficult to realise in any manufacturable assembly whilst 
successfully containing the fluid.

\begin{figure}[h] 
    \centering
    \textbf{FLUIDIC DAMPING CONCEPT}
    \caption{D3---Fluidic damping mechanism.}
    \label{fig:fluidic}
\end{figure}

\subsection{Isolation Mechanisms}

Early concepts made use of linkages to transform vertical and lateral 
oscillations into lateral oscillations along damped linear bearings, which 
would have been most easily realised using a viscoelastic damper. 
Figure~\vref{fig:concept1} details one such embodiment.

\begin{figure}[h] 
    \centering
    \textbf{ELOISE'S LINKAGE CONCEPT}
    \caption{C1---Simple linkage mechanism.}
    \label{fig:concept1}
\end{figure}

Some components of vertical vibrations were not damped by the mechanism in 
Figure~\ref{fig:concept1}; these were transmitted to the base of the mount such 
that the mount was not completely isolated. Repeating the mechanism by using 
multiple layers of linkages was explored in the concept detailed in 
Figure~\vref{fig:concept2}. Each layer damps a fraction of the vertical 
oscillations transmitted downwards by the layer above resulting in more 
isolation.

\begin{figure}[h]
    \centering
    \textbf{RUSSELL'S LINKAGE CONCEPT}
    \caption{C2---Layered linkage mechanism.}
    \label{fig:concept2}
\end{figure}

However, the fundamental issue with linkage based isolaton mechanisms is the 
feature size of those linkages. Thin members are prone to resonance with large 
amplitudes at relatively lower frequencies \cite{citation needed}, possibly 
within the audible spectra (20 Hz--20 kHz).

Spherical bearings also featured in design concepts. Figure~\vref{fig:concept3}
describes one such embodiment where a layer of spherical bearings were
sandwiched between two platforms. The upper platform was to be made of a hard 
material such as stainless steel or tungsten carbide, in order to efficiently 
transmit vibrations through to the bearings via point contacts. The bearings 
allowed the top platform to move with the equipment in three degrees of 
freedom: two lateral directions and lateral rotation. The bottom platform would 
have been made of some viscoelastic material with holes for the bearings to 
rest in. Both vertical and lateral oscillations would have been damped by this 
platform and dissipated as heat.

This concept had a major drawback: the damping mechanism was not confined to a 
single degree of freedom. It would have been challenging to determine a single 
damping factor and resonant frequency. The isolation mechanism was an anomaly 
in the concept generation phase, as the damping mechanism was not 
interchangable with those in Section~\vref{sec:damping_mechanisms}.

\begin{figure}[h]
    \centering
    \textbf{GEORGE'S PLATFORM CONCEPT}
    \caption{C3---Viscoelastic platform concept.}
    \label{fig:concept3}
\end{figure}

Isolation cones are a popular type of mount, an example is Nordost's Sort 
Kones\textregistered. The Sort Kones are made of three parts: a post, spherical 
bearing, and cone. The bearing rests on the top of the post and the cone sits 
on top of the bearing, such that the cone is free to swivel and rotate. 
However, cones seek only to isolate the mounted equipment and allow free 
movement---they do not attenuate vibrations.

Figure~\vref{fig:concept4} demonstrates how an attenuating mechanism could have 
been incorporated into the post of an isolation cone. The spring in the 
isolation mechanism could have been replaced with one of the damping mechanisms 
in Figure~\ref{fig:viscoelastic} and Figure~\ref{fig:magnetic}.


\begin{figure}[h]
    \centering
    \textbf{MAT'S CONE CONCEPT}
    \caption{C4---Damped cone concept.}
    \label{fig:concept4}
\end{figure}

An alternative mechanism was devised to translate horizontal and lateral 
oscillations into oscillations of a spherical bearing along a race. 
Figure~\vref{fig:concept5} details the geometry required to achieve this 
translation.

\begin{figure}[h]
    \centering
    \textbf{RUSSELL'S POINTY BOTTOM CONCEPT}
    \caption{C5---Cone and bearing concept.}
    \label{fig:concept5}
\end{figure}

When no force is applied to the top piece, the bearing would travel to the 
bottom of its race. Increasing the load would cause the bearing to move up the 
race due to the gradient of the curve on the top piece, which was greater than 
the gradient of the race. There would be some component of the reaction force 
acting on the bearing perpendicular to the curve which points up the the slope 
of the race. However, the gradient of the curve decreases travelling up the 
slope, so for a given load the component of the reaction force would also 
decrease. Therefore, for each load, there is a different equilibrium position 
somewhere along that race where the components of the bearing's weight and the 
reaction force acting on the bearing in the direction of the race are equal and 
opposite. The mechanism acts as a geometric spring with some stiffness.

The frictional sheer forces in the mechanism would provide significant damping 
due to the magnitude of the reaction forces involved. Furthermore, one of the 
damping mechanisms in Figure~\ref{fig:viscoelastic} or 
Figure~\ref{fig:magnetic} could be coupled to the linear motion of the 
bearings to tune the system to a desired natural frequency and damping 
factor---this could result in quite a large footer.

A drawback of this concept is difficulty in containing the bearings, the user 
would be free to remove the top piece, and the bearings would all come out of 
their races. Competitors could easily see how the mechanism works and replicate 
it for themselves.

Figure~\vref{fig:concept6} details a variation of C5, whereby the bearings 
travel outwards up the slope. This creates more space for a damping mechanism 
for each race around the outside of the footer. A fluid could be contained in 
the cavity in the bottom piece to damp vibrations using the mechanism described 
in Figure~\ref{fig:fluidic}.

\begin{figure}[h]
    \centering
    \textbf{RUSSELL'S POINTY TOP CONCEPT}
    \caption{C6---Alternative cone and bearing concept.}
    \label{fig:concept6}
\end{figure}

The flexibility of the alternative cone and bearing concept C6, coupled with 
its innovative design set the concept apart from its alternatives. In 
\cite{maguire2016vibration}, it was shown how concept C6 with damping mechanism 
D2 best matched the customer's expectations and engineering requirements. This 
is the concept which informed the next stage of development and our final 
design.

\section{Design Development} 
\subsection{Opposing Magnets}
\subsubsection{Magnet Simulation}

As the footer houses 12 rare earth magnets, considerations have to be made for 
the way in which these magnets interact with the surroundings of the footer as 
they could cause damage or interfere with mounted or surrounding 
equipment footer. Using a software package called 
MagNet\textsuperscript{\textregistered} produced by 
infolytica\textsuperscript{\textregistered}, it was possible to perform a 2D 
finite element analysis of the footers, allowing the effects of these magnets 
to be modelled. The analysis performed in this section is for the geometry of 
the final design, but the same procedure was performed following each 
significant revision during the design process. When modelling the effect of 
these magnets the two extreme cases were considered: at minimum and maximum 
separation as detailed in Figure~\ref{figure:minsep} and 
Figure~\ref{figure:maxsep}.

\begin{figure}[h]
    \centering
    \includegraphics[width=\textwidth]{Images/maxpreload.png}
    \caption{Flux state with minimum separation of magnets}
    \label{figure:minsep}
\end{figure}

\begin{figure}[h]
    \centering
    \includegraphics[width=\textwidth]{Images/minpreload.png}
    \caption{Flux state with maximum separation of magnets}
    \label{figure:maxsep}
\end{figure}

It can be seen that the flux density outside of the footer is negligible as it 
has a value of the order $10^{-5}~\si{\tesla}$ meaning that there would be 
minimal to no interactions between the footer and its surroundings as nearly 
all of the magnetic flux is contained within the core of the device.

\subsection{Preloading Mechanisms}
\subsection{Containment}
\subsection{Dynamic Analysis}
\subsubsection{Parametric Model} \label{sec:parametric-model}
\subsubsection{System Tuning}
\subsubsection{Experimentation}

To model the effectiveness of the product, sample data was collected to act as 
an input for the model; this data was used to simulate the driving force in the 
characteristic equation previously stated (Equation \ref{eq:2ndDiff}).  In 
order to obtain this information, an experiment was completed.

A 5V signal was generated with a PicoScope (pico Technology, PicoScope 2204A) 
and fed through an accelerometer (STMI electronics LIS344ALH); this 
accelerometer was coupled to the top of an amplifier connected to a standard AC 
mains power supply, nominal voltage of 240V at a frequency of 50Hz. Using the 
pico Technology software, data from the accelerometer was captured for three 
conditions; the amplifier turned off, the amplifier turned on, and the 
amplifier turned on sitting on four half squash balls placed under the footers. 
Squash balls are a rudimental method of noise damping and were tested to 
compare this known method against the information acquired from the 
mathematical model.

The data captured came in the form of a voltage output, to determine the 
acceleration from this information it was necessary to form a transformation 
equation. To do so, the technical data sheet for the accelerometer was found 
and in it there was clear conversion data which lead to the following

\begin{equation} \label{eq:acceleration}
a  =  (0.66 V - 1.65) 9.81
\end{equation}

as derived from \cite{SensorManual}

Where a is the acceleration $(ms^{-2})$, V is the voltage response from the 
accelerometer (V).

With the acceleration response and the mass of the amp, the driving force from 
the vibrations was calculated (Equation \ref{eq:Newton}) for each of the 
conditions.

Firstly, the data for the amplifier off, the control, was compared to the 
amplifier on. Both sets of information were processed and their spectra plotted 
as shown in Figure \ref{fig:MatGraphs1}

\begin{figure}[h]
    \centering
    \includegraphics[width=\textwidth]{Images/MatGraphs}
    \caption{Compared Frequency Responses when Amplifier off and on}
    \label{fig:MatGraphs1}
\end{figure}

The amp off data has a thin peak at 50Hz; this is due to the UK mains power 
supply rated at 50Hz. Although the amplifier was turned off it was still 
connected to the mains and therefore it is not unusual to witness a peak. With 
the amplifier on the fundamental frequency increases in power significantly up 
to -40dBV from -130dBV. The harmonics can be seen at 50Hz intervals although 
the general noise of the signal remains the same. Next, the amplifier on data 
was compared to the squash ball damped data to show how rudimental damping will 
affect the spectra and indicates how to determine successful damping for the 
dynamic model. The graphs are shown in Figure \ref{fig:MatGraphs2}

\begin{figure}[h]
    \centering
    \includegraphics[width=\textwidth]{Images/MatGraphs2}
    \caption{Compared Frequency Responses with and without Squash Balls}
    \label{fig:MatGraphs2}
\end{figure}

From the undamped spectra, it can be observed that the fundamental frequency is 
50Hz with a peak at around -50dBV and harmonics at 50Hz intervals. There is 
heavy noise on the spectrum which will need to be reduced to provide an 
improved sound quality. From the simple damping the squash balls provide this 
noise has been significantly attenuated, along with this success the harmonics 
have consistently lower peaks whilst the fundamental frequency retaining a 
similar power to the undamped spectrum. This provides an example of successful 
damping which can be used as a comparison to the later mathematically damped 
data. The aim is to further reduce the noise and lower the power of the 
harmonics as much as possible, preferably below the noise floor which can be 
observed to be around -130dBV.

\subsubsection{Dynamic Simulation}

It was determined that a dynamic model of the system would be necessary to 
quantify the potential success of the product. This could visually display the 
effects of the footers on attenuation of a signal from a HiFi system, and 
hence establish whether the sound quality had been improved. 

The system can be simplified to a spring mass damper system---a driving force 
in the vertical direction $F_{dy}$ is supplied by the vibration of the HiFi 
system and the opposing forces from the magnetic repulsion force $F_{my}(y)$ 
and damping coefficient $c$. Considering these components, it was possible to 
develop a second order differential equation which when solved would 
characterise the function of the footers.
\begin{equation} \label{eq:2ndDiff}
    F_{dy} = my^{\prime \prime} + cy^\prime + F_{my}(y) 
\end{equation}
where $m$ is the mass of the system and $y$ is the displacement due to 
vibrations.

Three components were to be found to complete the model: $c$, $F_{dy}(y)$, and 
$F_{my}(y)$. Ideally the damping coefficent for various masses would be 
determined empirically, however, using the theoretical approach outlined in 
Section~\ref{sec:parametric-model} the damping coefficient could be found given 
the dynamic friction coefficient $\mu$ between the tungsten carbide bearing and 
the stainless steel race. For these materials $0.4 < \mu < 0.6$. Substituting 
the limits into Equation~\ref{?}, the damping coefficient was deduced to lie in 
the range $158.7182~\si{\newton\second\per\meter} < c < 
238.0773~\si{\newton\second\per\meter}$.

The value of the magnetic repulsion force can be calculated, as outlined in 
section XXXX using equation XXX. The repulsion force is driven by the 
horizontal movement of the ball bearings; this motion is directly proportional 
to the vertical displacement from vibrations.

Simulink was chosen to drive the dynamic model due to the straight-forward 
construction of differential equations it provides; it offers a visual 
representation of the system and works hand-in-hand with MatLab allowing for 
simple data processing. Simulink computes an iterative numerical solution and 
outputs a corresponding time series to the MatLab workspace.

The first step to modelling was to consider each element of the characteristic 
equation and form a series of blocks to represent them. The output from MatLab 
represented the acceleration from the accelerometer ($y^{\prime \prime}$) 
therefore two \textquotesingle integrator\textquotesingle blocks were used in 
series to provide branches for $y^{\prime}$ and y. Each of these branches were 
then routed and manipulated to determine their effects on the equation.
Figure \ref{fig:MatModel} displays the final form of the model.

\begin{figure}[h]
    \centering
    \includegraphics[width=0.8\textwidth]{Images/MatModel}
    \caption{Compared Frequency Responses with and without Squash Balls}
    \label{fig:MatModel}
\end{figure}

\textquotesingle SimIn\textquotesingle provides the experimental data from the 
MatLab workspace and similarly, \textquotesingle SimOut\textquotesingle exports 
the mathematically damped data; \textquotesingle Scope\textquotesingle displays 
the input and output data together in the time domain. The triangular blocks 
multiply the branch by a constant stored in MatLab; \textquotesingle 
K\textquotesingle is a subsystem modelling the equation for magnetic repulsion 
force for a given displacement, y. These branches summed together as shown by 
the circular block will simulate the differential equation and hence the 
damping from the footers.

The driving force calculated from the experimental data was used as the input 
for the dynamic model; the value of damping coefficient is constant and the 
magnetic damping varies as the displacement of the data changes. A Fourier 
Transform was applied to the output of the model to analyse the spectrum. The 
\textquotesingle Amp on\textquotesingle data is used as the input for the 
system and Figure \ref{fig:CoF} displays the output spectra for the values of 
damping coefficient at the lower and upper bound values of coefficient of 
friction.

\begin{figure}[h]
    \centering
    \includegraphics[width=\textwidth]{Images/MatGraph3}
    \caption{Power Output for Coefficient of Friction Extremes}
    \label{fig:CoF}
\end{figure}

In both graphs, the fundamental frequency can no longer be observed; a peak at 
around 30Hz is apparent however its low power and frequency that differs from 
50Hz indicate that this is just noise. The noise is effectively identical in 
both cases and hence it can be concluded that the difference in damping 
coefficient between the 0.2 range of coefficient of friction is negligible. For 
further analysis, the median coefficient of friction will be used, 0.5; this 
returns a damping coefficient of 198.3978Ns/m.

The amp on data is displayed with the mathematically damped spectrum for 
comparison between the input and output of the dynamic model in 
Figure~\ref{fig:DampedVUndamped}.

\begin{figure}[h]
    \centering
    \includegraphics[width=0.8\textwidth]{Images/MatGraph4}
    \caption{Comparison of Undamped and Damped Response}
    \label{fig:DampedVUndamped}
\end{figure}

The red curve shows the mathematically damped data; the fundamental frequency 
and its harmonics are indistinguishable from the noise. The entire signal has 
been reduced to below the noise floor of around -140dBV. The noise has been 
almost completely attenuated.
The success of the footers is likely to be less significant than that of the 
model. The primary reason for this discrepancy is the value of damping 
coefficient used; the damping factor was simply estimated and the damping 
coefficient calculated mathematically and assumed to be constant. Ideally, this 
would be found empirically; a series of prototypes would be produced and a 
similar experiment carried out to find the true damping coefficient. However, 
it has not been possible to do so and hence a mathematical estimation used 
instead. 
If the footers have a fraction of the successs they have been predicted to 
have, there will be a significant improvement in sound quality. 

%---------------------------------------------------------------------------------------------------------------------------------------------------------------------------------------------------------
% MATERIAL CONSIDERATIONS - LAUREN
%---------------------------------------------------------------------------------------------------------------------------------------------------------------------------------------------------------

\subsection{Material Considerations}
\subsubsection{Neodymium Magnets}
Neodymium magnets are the most widely used type of rare-earth magnet, are the 
strongest type of permanent magnet commercially available and are of greatest 
benefit in applications where there is limited space. The magnetic properties 
termed remanence, coercivity and energy product dictate the performance of 
neodymium magnets. They measure the strength of the magnetic field, the 
material?s resistance to becoming demagnetized and the density of magnetic 
energy respectively. Given that there are numerous grades available, an 
overview of the range of values are outlined in Table 1. [1] The grade chosen 
for the project was N45SH, coated with three layers (nickel, copper and nickel) 
to reduce corrosion and provide a smooth finish. The brittleness of the magnets 
was also compensated for such that it does not undergo shear force during the 
cyclic loading. 

\subsubsection{Tungsten Carbide Spherical Bearings}
Tungsten carbide possesses Young?s modulus 530 GPa, more than two times of that for the grade of stainless steel chosen (190 ? 203 GPa) [4] [5]. Due to Hooke?s law, the material does not undergo large deflections which in turn ensured efficient energy transfer between the Hi-Fi components. The bearings do not require any lubrication as they grind down any particulates that may enter the system.

\subsubsection{Stainless Steel and Aluminium Framework}

This improved the mount?s damping capability because the material essentially 
acted as a Faraday cage with high resistance. Stainless steel is totally 
recyclable and together with its high resistance to corrosion, greatly 
increased the lifetime and hence sustainability of the product. However, it was 
decided that the mount, apart from the top piece, core and spike, should 
instead be machined with aircraft grade aluminium (6061-T6) due to the steep 
manufacturing costs associated with stainless steel. These parts were excluded 
because they were subjected to the greatest stress. Aluminium is cheaper to 
manufacture but is extremely lightweight, a far better conductor (36.9  106 
Siemens/m), widely available and recycled and can be easily extruded at low 
cost to produce the complex geometries present. [6]

%dimension selection- bearings driving dimensions, tracks, screws

%Pre-load system- maths behind pre-load weights, check torque manageable, effect of pre-loading on mass and nat freq & damping factor, magnet dimensions,bearings

%selections of materials- steel durable, contain flux


%(Describe the development of the design to
%determine say the principal dimensions or weights,
%selection of materials)

%---------------------------------------------------------------------------------------------------------------------------------------------------------------------------------------------------------
% DETAIL DESIGN - RUSSELL
%---------------------------------------------------------------------------------------------------------------------------------------------------------------------------------------------------------

\section{Detail Design} \label{sec:DetailDesign}

%(Describe the minor details, such as nuts and bolts, - RUSSELL
%details for avoiding stress concentrations suitability of screws,-GEORGE specify any bought-in
%components.)

%---------------------------------------------------------------------------------------------------------------------------------------------------------------------------------------------------------
% DFM /DFA- ELOISE%---------------------------------------------------------------------------------------------------------------------------------------------------------------------------------------------------------
 
\section{Design for Manufacture} 

DFM defines the design/engineering of a product so as to best optimise its quality whilst minimising its cost to manufacture\cite{dfm}. Factors affecting this cost include the number of off-the-shelf and machinable parts, the set-up time of required machinery, the material type, dimensional tolerances as well as secondary processes. Generally, a compromise is reached between the functional quality of a product and the cost of manufacture however, considering the current extortionate pricing of similar existing products, certain design choices have taken precedence over their implications in a manufacturing context, for the example the complex central piece (see drawing COR080-0003).

\subsection{Off-The-Shelf Parts}
Due to the complex geometry required from our design, only few components may be bought in, namely (per mount);  six �10mm tungsten carbide ball bearings, six \diameter 10 x 1.5 mm neodymium magnets, six M4 by 20 mm AISI A2 steel countersunk Torx security screws and one M4 by 20 mm countersunk hex socket with partial thread. These components are readily available excluding the tungsten carbide ball bearings which must be sourced from a specialist supplier.
Table~\ref{table:BoM} details potential sources for the aforementioned parts and their corresponding cost.


\begin{table}[h]
	\centering
	\caption{Bill of Materials}
	\begin{tabular}{@{}rp{9em}p{9em}rrr@{}}
		\toprule
		Item No. & Part No. & Description & Qty. & �/Piece & � Total \\
		\midrule
		1 & COR080-0003 & Isofonics core & 1 & --- & --- \\
		2 & PLD080-0003 & Isofonics preloading crown & 1 & --- & --- \\
		3 & PLD010-1004 & Isofonics preloading backstop & 6 & --- & --- \\
		4 & F669-N45SH-10 (first4magnets) & 10 x 1.5mm neodymium magnet & 12 & 0.30 & 3.6  \\
		5 & 10MMTUNGSTENB ALLS (VXB Bearings) & 10mm tungsten carbide ball & 6 & 20.30 & 243.60 \\
		6 & TOP080-0004A & Isofonics top piece & 1 & --- & --- \\
		7 & RET080-0003 & Isofonics retainer & 1 & --- & --- \\
		8 & M4X20-CSK-ST (westfieldfasteners) & Partially threaded M4 x 20mm c'sunk security screw & 1 & 0.09 & 0.09 \\
		9 & M4X20-CSK-H (westfieldfasteners) & Partially threaded M4 x 20mm c'sunk hex socket screw & 1 & 0.04 & 0.04 \\
		10 & USR080-0001 & Isofonics removable spike & 1 & --- & --- \\
		11 & RET080-1002 & Isofonics retainer bottom & 1 & --- & --- \\
		12 & USR080-1002 & Isofonics removable base & 1 & --- & --- \\
		\bottomrule
\end{tabular}
\label{table:BoM}
\end{table}


\subsection{Machined Parts}
All other parts may be machined using a 3 axis CNC mill excluding the central piece (see drawing COR080-0003), which requires 4 axis machining. 4 axis machining is substantially more expensive but allows for more intricate designs by using a 4th axis to reposition the part between 3 axis operations, a feature required for this design\cite{cnc}.

\subsubsection{Set-up Time}
Cost is substantially driven by time; time to remove material in the machining process as well as set-up time of the machine itself\cite{SetUp}. Conveniently, our design is highly symmetrical meaning time and money is saved through lacking the need of a complex orienting mechanism given that the part's orientation prior to machining is irrelevant.

\subsubsection{Material Type}
Steel is harder, and therefore more expensive, to machine than softer materials however, its properties make it invaluable to the quality of the product (see Section~\ref{sec:DetailDesign}) and pose no problem for the advanced machining capabilities of today. The product's entire composition out of steel was initially proposed but after receiving a quote from 'Sylatech' (r/c), a machining company in the northeast of England for �---- per piece, it was evident some optimisation was required to quote a retail price for our product that allowed for a 35\% \cite{ProfitMargins} profit margin.For this reason, it was decided that all non-critical parts may be machined from aluminium, an alternative roughly 1.5 times cheaper\cite{AluVSteel} than steel. 'Non-critical' in this context refers to all parts excluding the complex central piece (COR080-0003) for which steel is required to contain magnetic flux leakage and remain unaffected by frictional effects of the moving bearings. Additionally, aluminium is roughy 5 times easier to machine than stainless steel\cite{AluVSteel} saving significantly on time, and therefore cost.

\subsubsection{Tolerances}
Initially all parts were to be machined at fine linear and angular tolerances of +/-0.1mm and +/-$1\,^{\circ}\mathrm{}$ respectively to ensure the overall quality of the product. However, in the interest of reducing cost, it was concluded that the rails contained in the core piece are the only parts to be machined to a high tolerance since they are to fit plush to the bearings; all other parts may be machined to a lower, and therefore cheaper, tolerance of +/- xmm.

\subsection{Secondary Processes}
\subsubsection{Finishings}

A 2J finish (brushed finish) has been chosen as it is cheaper to produce than polishing and is practical in that it is resistant to scratches whilst being aesthetically pleasing.

\subsection{Volume}
Finally, the volume of production is of great importance; with too small a batch size, set up costs and jig production costs become impractical and with too large a batch, storage costs pose a problem. Considering the high cost and exclusive appeal of this product, low volume batch production in the order of 100 is appropriate so as to eliminate any unnecessary storage costs and allow adequate response to market needs.

\subsection{Manufacture Costs} \label{sec:ManCosts}
%subtract material cost from quote for maching costs
Table \ref{table:BoM} details the cost of off-the-shelf parts and machining per mount, totalling �y. It is worth noting that the off-the-shelf pricing has been quoted for relatively low volume purchases, specifically within the range 100-200 pieces, and would decrease for larger scale production as to be used for this product; a batch volume 600 of pieces.


\subsection{Optimisation}
 %the core may be machined in separate parts that are later joined to avoid the complexity of machining the reverse chamfers as shown in Figure~\ref{fig:core}.

\begin{figure}[h]
   \centering
   \includegraphics[width=0.3\textwidth]{Images/core}
   \caption{Core}
   \label{core}
\end{figure} 


%what tools are required - allen (purple engraved) , 
%can the design be altered slightly to
%improve convenience and hence cost of
%construction? - smaller less tracks (although crisitcal size exists), split base solid part into separate parts to join, difficults part from reverse chamfers so have one solid base and join blobs.
%parts- 1 top, 1 spike, 12 magnets, 1 retianer, 6 M4 by 20mm countersunk torx security screws, 1 M4 by 20mm countersunk hex socket with partial thread, 1 insert, 1 bottom,1 base, 6 adjustables per batch, jig, jig sleeve
%Can a few common components be used rather than many different varieties?- no, high budget
%
\section{Design for Assembly} 
Due to the small scale of our product, hand assembly is required. Considering the complexity of the design and physical impracticality of overcoming the repulsive magnetic forces during its assembly, a jig has been designed, as shown in drawings JIG080-0001 and JIG080-1001, with full accompanying assembly instructions (see Isofonics Assembly Instructions ISO080-INS).

\begin{figure}[h]
    \centering
    \subfloat[Assembly Jig Base]{{\includegraphics[width=0.4\textwidth]{Images/jigbase} }}%
    \qquad
    \subfloat[Assembly Jig Sleeve]{{\includegraphics[width=0.3\textwidth]{Images/jigsleeve} }}%
    \caption{Assembly Jig Components}
    \label{fig:JigAssembly}
\end{figure}

\subsection{Assembly Costs} \label{AssemCosts}
Considering the assembly jig is to be re-used indefinitely, its cost is negligible within the context of the product's assembly. Assembly costs are then only defined by the time taken to manually assemble the device and the cost of manual labour which can be estimated at 2 minutes and \textsterling8/hr respectively. This equates to a total of \textsterling0.27 per piece.


%---------------------------------------------------------------------------------------------------------------------------------------------------------------------------------------------------------
% DESIGN FOR SUSTAINABILITY - MAT
%---------------------------------------------------------------------------------------------------------------------------------------------------------------------------------------------------------
\section{Design for Sustainability} 
%machined material, scrap of which can be melted purely
%scrap from machinery is recyclable
%possibility to carbon offset the production of products given extortionate pricing 
%longevity of product 

%Have you considered using sustainable materials
%or processes in your design? Can you make it out
%of recycled, novel, or plant-based materials? Can
%it be recycled itself?

%---------------------------------------------------------------------------------------------------------------------------------------------------------------------------------------------------------
% COMMERCIAL CONSIDERATIONS - ELOISE/LAUREN
%---------------------------------------------------------------------------------------------------------------------------------------------------------------------------------------------------------

\section{Commercial Considerations} 
%Have you included a full Bill of Materials? (george)
%What are the major costs of materials and components? labour and tooling need carbide tools for steel (eloise)
What does it cost to machine parts- (Eloise)
% What is the life of the product before a new model has to be introduced and how many units can be sold in that
%period?- tracks and cone, lifetime magnets, on lubricant, pre-loading

 \subsection{Brand Development and Competitor Analysis}
 With the intention of establishing the design in the market, a brand identity was essential. Through several brainstorming sessions, the team agreed on the name ?Isofonics.? It was important to review the current market leaders, Stillpoints and Nordost, with their relevant products. Nordost primarily manufactures Hi-Fi audio cables but recently introduced additional products for resonance and power control. Namely, their ?Sort Kone,? acts as a vibration drain with a mechanical diode effect to prevent external vibrations from travelling up through the cone. Similarly, 3 cones are recommended per device but they offer 3 types of cone, each made from different materials. The most expensive is made from titanium and utilises ceramic bearings, retailing at �349.99 per cone. The 3 mounts do not necessarily have to be the same and the user may find it more beneficial to use a mixture. (See Figure 3) [7]
 
 \subsection{Stillpoints Patent Check}
 
 The Stillpoints patent details the design of a \lq{device for the control of vibrations comprising a retainer resting on a base and a plurality of bearings disposed within the retainer}\rq \cite{Stillpoints}within which \lq{the bearings are arranged in at least two layers}\rq \cite{Stillpoints}. It alludes to multiple alternative designs including an \lq{embodiment}\rq in which magnetic fields are employed, however differently to the usage detailed in our design. After studying the official claims made at the end of the patent it is concluded that in not featuring layers of bearings and using opposing magnetic fields as the main damping method, our design differs significantly from the design and alternatives claimed by Stillpoints.\\
 \textbf{It is worth noting this is a US patent and is not filed with the European Patent Office and so only stands to limit sales within the US}
 
 \subsubsection{Parallels with Stillpoints}
 The following details the main claim made by Stillpoints and highlights areas of concern:

\lq{A device for the control of vibrations comprising:
\textbf{a retainer}, the retainer constructed and arranged to rest upon a base, at least a \textbf{portion of the base defining a substantially flat surface}; and \textbf{a plurality of bearings disposed within the retainer, the bearings arranged in a first layer and a second layer}, the second layer disposed on the first layer, the first layer
comprising three or more bearings, and the second layer comprising at least one bearing, each bearing in
the first layer constrained on its bottom by only the substantially flat surface of the base, on its side by the
retainer, the bearings in the first layer supporting the at least one bearing in the second layer, the retainer defining a surface which is in substantially tangential contact with the bearings in the first layer}\rq\cite{Stillpoints}.

It is clear that the main claim focuses on the use of layers of bearings, in this way, our design is fundamentally different. Since our design differs from that outlined in the main claim, all further claims based on the first are irrelevant. An element of originality of our design lies in the use of magnets which is not mentioned at all within the official claims section of the patent.

The use of magnetic fields is mentioned during the preamble and is as follows:

\lq{In some embodiments the first base member 119 and second base member 123 may have opposing magnetic fields.}\rq\cite{Stillpoints}

\begin{figure}[h]
    \centering
    \subfloat[Stillpoints Mechanism Diagram]{{\includegraphics[width=0.4\textwidth]{Images/Stillpoints} }}%
    \qquad
    \subfloat[Group 25 Mechanism Diagram]{{\includegraphics[width=0.4\textwidth]{Images/Group25} }}%
    \caption{Stillpoints Mechanism Comparison}
    \label{fig:Patent}
\end{figure}
As can be seen from Figures 1 and 2 , this comment is entirely unrelated to the use of opposing magnetic fields to essentially replace the spring mechanism shown above and consequently does not affect our design.

Figure ~\ref{fig:Patent}

\subsubsection{Points of Originality}
-Use of \textbf{opposing magnetic fields} for lateral damping as opposed to springs\\
- \textbf{Cone with retaining flange} replacing layers of bearings\\
-\textbf{Pre-loading mechanism} in which a screw in tension moves a fixed magnet decreasing its distance from the bearing magnet thus increasing the strength of the system (adjustable for a variety of masses)

\subsection{Marginal Costs}
With respect to production costs, margins of 100\% for product distribution, 100\% for retail and 35\% for profit are to be met. Given that machining and assembly costs total �x (see section \ref{sec:ManCosts}) and �y (see section \ref{sec:AssemCosts}) per piece respectively, the product's cost of production is �r per piece. The aforementioned product margins then dictate a retail price of �(1.35 * r).


%---------------------------------------------------------------------------------------------------------------------------------------------------------------------------------------------------------
% DISCUSSION - LAUREN
%---------------------------------------------------------------------------------------------------------------------------------------------------------------------------------------------------------
\section{Discussion} 
\subsection{Specification Fulfilment}
\subsubsection{User Requirement Specification Tree}
With reference to the URS tree, the final proposal incorporated a cone with a retaining flange which held the magnets in place whose separation was altered in order to accommodate a range of masses. The four degrees of freedom were thus defined by all translational motion and free rotation. As described earlier in the report, the magnet behind the bearing is moved while the other is fixed and the preloading mechanism allowed the user to adjust the position of the fixed magnet. Namely, the crown element moved the back stops for the fixed magnet inward as a single preloading screw was tightened. 

The requirement for a rigid load path was integral to the performance of the damping mechanism. The interaction between the magnets and the spherical bearings in conjunction with the hardness of their material clearly diminished the excess mechanical vibrations generated by microphonic effects as demonstrated by the graphs produced by the dynamic model. However, the data yielded during the experiment which was then inputted to the simulation, did not totally reflect the execution of the design. Employing the squash balls as a method of dampening the noise produced by the amp did validate the concept but this could be supported further by utilising a prototype and later determining the damping coefficient for different loads empirically. If these results demonstrated a fraction of the model?s success, then the mount can still be confidently viewed as a success. 

The Hi-Fi equipment must only rest on 3 footers to prevent further resonances and heating in the arrangement. If another was used, it would either prove redundant or induce further noise. The cone provided a single point of contact which in turn meant that 3 footers were needed to support one piece of equipment. Partnered with the large base plate, this guaranteed that the mounts would not slip and the system was amply stable. For this reasoning, the footers would be sold in sets of three.

A 2-D finite element analysis software was used to model the magnetic flux distribution within the mount. Although it was concluded that there was no additional interference or damage to the Hi-Fi system, ideally a 3-D finite element analysis software would better determine whether the separation distance or the grade of magnet needed to be amended. 
%Discuss the advantages of the proposed design in comparison with alternatives. Discuss how well the design meets the original specification. - Lauren, refer to tree

%---------------------------------------------------------------------------------------------------------------------------------------------------------------------------------------------------------
% DEVELOPMENTS - LAUREN
%---------------------------------------------------------------------------------------------------------------------------------------------------------------------------------------------------------

\subsection{further developments}
There is great potential for further work and like Nordost, a family of products adapted to support heavier devices such as loudspeakers could also be developed by changing the material for the spherical bearings or by varying the magnetic field strength to yield a new damping factor and related coefficient. A companion phone application which enables the user to adjust the preload for a desired condition could also replace the preloading graphs provided by the parametric model. Machining could be improved by optimising tolerancing for the tracks and will thus increase profit margins and set the product aside from the mentioned competitors. 
%Discuss potential further developments. -
%Discuss the impact on the design of new materials - cheaper with aluminium
%Discuss uncertainties and assumptions in the design process and indicate what practical measurements might be required to give the necessary data to improve the design. - measure damping coefficient, 3d modelling

%---------------------------------------------------------------------------------------------------------------------------------------------------------------------------------------------------------
% CONCLUSION - GEORGE
%---------------------------------------------------------------------------------------------------------------------------------------------------------------------------------------------------------
\section{Conclusion} 
%The conclusions should be a set of brief
%statements of the main results of the project.
%Sometimes bullet points can be used to give
%emphasis


%-----------------------------------------------------------------------------------------------------------------------------------
% REFERENCES
%-----------------------------------------------------------------------------------------------------------------------------------
\section{References}
\printbibliography
%-----------------------------------------------------------------------------------------------------------------------------------

\cleardoublepage
\appendix
\section{Project Plan and Members' Contributions} 
\includepdf[pages=-,fitpaper=true]{Images/Drawings/COR080-0003.PDF}
\includepdf[pages=-,fitpaper=true]{Images/Drawings/PLD010-1004.PDF}
\includepdf[pages=-,fitpaper=true]{Images/Drawings/PLD080-0003.PDF}
\includepdf[pages=-,fitpaper=true]{Images/Drawings/RET080-0003.PDF}
\includepdf[pages=-,fitpaper=true]{Images/Drawings/RET080-1002.PDF}
\includepdf[pages=-,fitpaper=true]{Images/Drawings/TOP080-0004.PDF}
\includepdf[pages=-,fitpaper=true]{Images/Drawings/USR080-0002.PDF}
\includepdf[pages=-,fitpaper=true]{Images/Drawings/USR080-1002.PDF}

\includepdf[pages=-,fitpaper=true]{Images/jig/JIG080-0001.PDF}
\includepdf[pages=-,fitpaper=true]{Images/jig/JIG080-1001.PDF}

\includepdf[pages=-,fitpaper=true]{Images/jig/ISO080-INS.pdf}



\end{document}
