\documentclass{article}
\usepackage{varioref}
\usepackage[style=ieee,backend=bibtex]{biblatex}
\usepackage{filecontents}
\begin{filecontents}{\jobname.bib}
    @book{meyers2009mechanical,
        title={Mechanical behavior of materials},
        author={Meyers, Marc A and Chawla, Krishan Kumar},
        volume={2},
        pages={98--103},
        year={2009},
        publisher={Cambridge University Press Cambridge}
    }
\end{filecontents}
\addbibresource{\jobname.bib}

\begin{document}
    
\section{Design Concepts}

During initial research it was noticed the most successful products on the 
market met three primary criteria:

\begin{itemize}
    \item They allowed the mounted equipment to move with four degrees of 
    freedom: three in translation and one in rotation.
    \item They used single degree of freedom systems for damping. Simpler 
    damping mechanism were more effective at consistently attenuating 
    unwanted frequency components.
    \item Vibrations were efficiently transmitted to the damping system from 
    the equipment. This was typically achieved using hard materials and limited 
    contact between the footer and equipment.
\end{itemize}

In summary, any concepts generated had to first \emph{isolate} the equipment 
then \emph{attenuate} vibrations by allowing the equipment to move freely and 
then damp vibrations in controlled manner.

In order to simplify the process of concept generation, the formulation of 
isolation and attenuation mechanisms was separated. Isolation mechanisms were 
designed to translate the four degrees of freedom in the movement of the 
equipment into the single degree of freedom of the damping mechanism. Those 
single degree of freedom damping mechanisms were designed to be easily applied 
to multiple isolation concepts.

\subsection{Damping Mechanisms} \label{sec:damping_mechanisms}

Three methods of damping were considered---not including the friction in 
bearings, which cannot be completely eliminated. These were viscoelastic, 
magnetic and fluid damping.

Viscoelastic materials, such as rubber or Sorbothane\textregistered, heat up 
when subjected to changes in mechanical stress \cite{meyers2009mechanical}. 
Hence, the amplitude of vibrations are reduced when the driving oscillations 
are transmitted through viscoelastic materials. Typically when these materials 
are used in footers, the vibrations are not limited to a single degree of 
freedom, so the effectiveness of attenuations depends on the vibration 
direction. This makes it difficult to model and specify a single system natural 
frequency or damping factor.

Figure~\vref{fig:viscoelastic} details how viscoelastic materials can be used 
in a system where vibrations have been translated to a single direction. A 
linear bearing compresses a viscoelastic sleeve on the same shaft.

\begin{figure}[h] 
    \centering
    \textbf{VISCOELASTIC DAMPING CONCEPT}
    \caption{Viscoelastic damping mechanism.}
    \label{fig:viscoelastic}
\end{figure}

Magnets can also be used to damp vibrations when constrained to a single 
direction. As a permanent magnet travels through a coil, an EMF is induced due 
to Faraday's law, which drives a current when the circuit is complete. The 
current flowing generates a magnetic field opposing the permanent magnet, 
according to Lenz's law. A possible assembly for this system is detailed in 
Figure~\vref{fig:magnetic}. The strength of magnets were to be tuned to the 
desired dynamic properties of the system.

\begin{figure}[h] 
    \centering
    \textbf{MAGNETIC DAMPING CONCEPT}
    \caption{Magnetic damping mechanism.}
    \label{fig:magnetic}
\end{figure}

Finally, a simple fluid damping system was considered, whereupon a bearing 
travels through some viscous fluid. Skin friction provides the damping force in 
Figure~\vref{fig:fluidic}. The size of the bearing could be tuned to meet the 
required constraints for damping factor and natural frequency. However, this 
concept was difficult to realise in any manufacturable assembly whilst 
successfully containing the fluid.

\begin{figure}[h] 
    \centering
    \textbf{FLUIDIC DAMPING CONCEPT}
    \caption{Fluidic damping mechanism.}
    \label{fig:fluidic}
\end{figure}

\subsection{Isolation Mechanisms}

Early concepts made use of linkages to transform vertical and lateral 
oscillations into lateral oscillations along damped linear bearings, which 
would have been most easily realised using a viscoelastic damper. 
Figure~\vref{fig:concept1} details one such embodiment.

\begin{figure}[h] 
    \centering
    \textbf{ELOISE'S LINKAGE CONCEPT}
    \caption{Concept 1---simple linkage mechanism.}
    \label{fig:concept1}
\end{figure}

Some components of vertical vibrations were not damped by the mechanism in 
Figure~\ref{fig:concept1}; these were transmitted to the base of the mount such 
that the mount was not completely isolated. Repeating the mechanism by using 
multiple layers of linkages was explored in the concept detailed in 
Figure~\vref{fig:concept2}. Each layer damps a fraction of the vertical 
oscillations transmitted downwards by the layer above resulting in more 
isolation.

\begin{figure}[h]
    \centering
    \textbf{RUSSELL'S LINKAGE CONCEPT}
    \caption{Concept 2---layered linkage mechanism.}
    \label{fig:concept2}
\end{figure}

However, the fundamental issue with linkage based isolaton mechanisms is the 
feature size of those linkages. Thin members are prone to resonance with large 
amplitudes at relatively lower frequencies \cite{citation needed}, possibly 
within the audible spectra (20 Hz--20 kHz).

Spherical bearings also featured in design concepts. Figure~\vref{fig:concept3}
describes one such embodiment where a layer of spherical bearings were
sandwiched between two platforms. The upper platform was to be made of a hard 
material such as stainless steel or tungsten carbide, in order to efficiently 
transmit vibrations through to the bearings via point contacts. The bearings 
allowed the top platform to move with the equipment in three degrees of 
freedom: two lateral directions lateral rotation. The bottom platform would 
have been made of some viscoelastic material with holes for the bearings to 
rest in. Both vertical and lateral oscillations would be damped by this 
platform and dissipated as heat.

This concept has a major drawback: the damping mechanism is not confined to a 
single degree of freedom, so would be very difficult to determine a single 
damping factor and resonant frequency. The isolation mechanism was an anomaly 
in the concept generation phase, as the damping mechanism was not 
interchangable with those in Section~\vref{sec:damping_mechanisms}.



\begin{figure}[h]
    \centering
    \textbf{GEORGE'S PLATFORM CONCEPT}
    \caption{Concept 3---viscoelastic platform concept.}
    \label{fig:concept3}
\end{figure}
    
\section{Detailed Design}
    
\end{document}