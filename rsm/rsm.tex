\documentclass[11pt]{article}
\usepackage{varioref}
\usepackage{wasysym}
\usepackage{booktabs}
\usepackage{xspace}
\usepackage{threeparttable}
\usepackage[hidelinks]{hyperref}
\usepackage[style=ieee,backend=bibtex]{biblatex}
\usepackage{filecontents}

\begin{filecontents}{\jobname.bib}
    @book{meyers2009mechanical,
        title={Mechanical behavior of materials},
        author={Meyers, Marc A and Chawla, Krishan Kumar},
        volume={2},
        pages={98--103},
        year={2009},
        publisher={Cambridge University Press},
        location={Cambridge}
    }
    @techreport{maguire2016vibration,
        title={Vibration Isolation and Attenuation Mounts},
        author={Maguire, R S and Hollands, E O and Jones, B A H and Miller, L R 
        and Hall, G A and Valladares-Hodder, M},
        institution={Durham University School of Engineering and Computing 
        Sciences},
        year={2016},
        month={12},
        type={Feasibility Report},
        location={Durham}
    }
\end{filecontents}
\addbibresource{\jobname.bib}

\begin{document}
    
\section{Design Concepts}

During initial research it was noticed the most successful products on the 
market met three primary criteria:

\begin{itemize}
    \item They allowed the mounted equipment to move with four degrees of 
    freedom: three in translation and one in rotation.
    \item They used single degree of freedom systems for damping. Simpler 
    damping mechanisms were more effective at consistently attenuating 
    unwanted frequency components.
    \item Vibrations were efficiently transmitted to the damping system from 
    the equipment. This was typically achieved using hard materials and limited 
    contact between the footer and equipment.
\end{itemize}

In summary, any concepts generated had to first \emph{isolate} the equipment 
then \emph{attenuate} vibrations by allowing the equipment to move freely and 
then damp vibrations in controlled manner.

In order to simplify the process of concept generation, the formulation of 
isolation and attenuation mechanisms was separated. Isolation mechanisms were 
designed to translate the four degrees of freedom in the movement of the 
equipment into the single degree of freedom of the damping mechanism. Those 
single degree of freedom damping mechanisms were designed to be easily applied 
to multiple isolation concepts.

\subsection{Damping Mechanisms} \label{sec:damping_mechanisms}

Three methods of damping were considered---not including the friction in 
bearings, which cannot be completely eliminated. These were viscoelastic, 
magnetic and fluid damping.

Viscoelastic materials, such as rubber or Sorbothane\textregistered, heat up 
when subjected to changes in mechanical stress \cite{meyers2009mechanical}. 
Hence, the amplitude of vibrations are reduced when the driving oscillations 
are transmitted through viscoelastic materials. Typically when these materials 
are used in footers, the vibrations are not limited to a single degree of 
freedom, so the effectiveness of attenuation depends on the vibration 
direction. This makes it difficult to model and specify a single system natural 
frequency or damping factor.

Figure~\vref{fig:viscoelastic} details how viscoelastic materials can be used 
in a system where vibrations have been translated to a single direction. A 
linear bearing compresses a viscoelastic sleeve on the same shaft.

\begin{figure}[h] 
    \centering
    \textbf{VISCOELASTIC DAMPING CONCEPT}
    \caption{D1---Viscoelastic damping mechanism.}
    \label{fig:viscoelastic}
\end{figure}

Magnets can also be used to damp vibrations when constrained to a single 
direction. As a permanent magnet travels through a coil, an EMF is induced due 
to Faraday's law, which drives a current when the circuit is complete. The 
current flowing generates a magnetic field opposing the permanent magnet, 
according to Lenz's law. A possible assembly for this system is detailed in 
Figure~\vref{fig:magnetic}. The strength of magnets were to be tuned to the 
desired dynamic properties of the system.

\begin{figure}[h] 
    \centering
    \textbf{MAGNETIC DAMPING CONCEPT}
    \caption{D2---Magnetic damping mechanism.}
    \label{fig:magnetic}
\end{figure}

Finally, a simple fluid damping system was considered, whereupon a bearing 
travels through some viscous fluid. Skin friction provides the damping force in 
Figure~\vref{fig:fluidic}. The size of the bearing could be tuned to meet the 
required constraints for damping factor and natural frequency. However, this 
concept was difficult to realise in any manufacturable assembly whilst 
successfully containing the fluid.

\begin{figure}[h] 
    \centering
    \textbf{FLUIDIC DAMPING CONCEPT}
    \caption{D3---Fluidic damping mechanism.}
    \label{fig:fluidic}
\end{figure}

\subsection{Isolation Mechanisms}

Early concepts made use of linkages to transform vertical and lateral 
oscillations into lateral oscillations along damped linear bearings, which 
would have been most easily realised using a viscoelastic damper. 
Figure~\vref{fig:concept1} details one such embodiment.

\begin{figure}[h] 
    \centering
    \textbf{ELOISE'S LINKAGE CONCEPT}
    \caption{C1---Simple linkage mechanism.}
    \label{fig:concept1}
\end{figure}

Some components of vertical vibrations were not damped by the mechanism in 
Figure~\ref{fig:concept1}; these were transmitted to the base of the mount such 
that the mount was not completely isolated. Repeating the mechanism by using 
multiple layers of linkages was explored in the concept detailed in 
Figure~\vref{fig:concept2}. Each layer damps a fraction of the vertical 
oscillations transmitted downwards by the layer above resulting in more 
isolation.

\begin{figure}[h]
    \centering
    \textbf{RUSSELL'S LINKAGE CONCEPT}
    \caption{C2---Layered linkage mechanism.}
    \label{fig:concept2}
\end{figure}

However, the fundamental issue with linkage based isolaton mechanisms is the 
feature size of those linkages. Thin members are prone to resonance with large 
amplitudes at relatively lower frequencies \cite{citation needed}, possibly 
within the audible spectra (20 Hz--20 kHz).

Spherical bearings also featured in design concepts. Figure~\vref{fig:concept3}
describes one such embodiment where a layer of spherical bearings were
sandwiched between two platforms. The upper platform was to be made of a hard 
material such as stainless steel or tungsten carbide, in order to efficiently 
transmit vibrations through to the bearings via point contacts. The bearings 
allowed the top platform to move with the equipment in three degrees of 
freedom: two lateral directions and lateral rotation. The bottom platform would 
have been made of some viscoelastic material with holes for the bearings to 
rest in. Both vertical and lateral oscillations would have been damped by this 
platform and dissipated as heat.

This concept had a major drawback: the damping mechanism was not confined to a 
single degree of freedom. It would have been challenging to determine a single 
damping factor and resonant frequency. The isolation mechanism was an anomaly 
in the concept generation phase, as the damping mechanism was not 
interchangable with those in Section~\vref{sec:damping_mechanisms}.

\begin{figure}[h]
    \centering
    \textbf{GEORGE'S PLATFORM CONCEPT}
    \caption{C3---Viscoelastic platform concept.}
    \label{fig:concept3}
\end{figure}

Isolation cones are a popular type of mount, an example is Nordost's Sort 
Kones\textregistered. The Sort Kones are made of three parts: a post, spherical 
bearing, and cone. The bearing rests on the top of the post and the cone sits 
on top of the bearing, such that the cone is free to swivel and rotate. 
However, cones seek only to isolate the mounted equipment and allow free 
movement---they do not attenuate vibrations.

Figure~\vref{fig:concept4} demonstrates how an attenuating mechanism could have 
been incorporated into the post of an isolation cone. The spring in the 
isolation mechanism could have been replaced with one of the damping mechanisms 
in Figure~\ref{fig:viscoelastic} and Figure~\ref{fig:magnetic}.

    
\begin{figure}[h]
    \centering
    \textbf{MAT'S CONE CONCEPT}
    \caption{C4---Damped cone concept.}
    \label{fig:concept4}
\end{figure}
    
An alternative mechanism was devised to translate horizontal and lateral 
oscillations into oscillations of a spherical bearing along a race. 
Figure~\vref{fig:concept5} details the geometry required to achieve this 
translation.

\begin{figure}[h]
    \centering
    \textbf{RUSSELL'S POINTY BOTTOM CONCEPT}
    \caption{C5---Cone and bearing concept.}
    \label{fig:concept5}
\end{figure}

When no force is applied to the top piece, the bearing would travel to the 
bottom of its race. Increasing the load would cause the bearing to move up the 
race due to the gradient of the curve on the top piece, which was greater than 
the gradient of the race. There would be some component of the reaction force 
acting on the bearing perpendicular to the curve which points up the the slope 
of the race. However, the gradient of the curve decreases travelling up the 
slope, so for a given load the component of the reaction force would also 
decrease. Therefore, for each load, there is a different equilibrium position 
somewhere along that race where the components of the bearing's weight and the 
reaction force acting on the bearing in the direction of the race are equal and 
opposite. The mechanism acts as a geometric spring with some stiffness.

The frictional sheer forces in the mechanism would provide significant damping 
due to the magnitude of the reaction forces involved. Furthermore, one of the 
damping mechanisms in Figure~\ref{fig:viscoelastic} or 
Figure~\ref{fig:magnetic} could be coupled to the linear motion of the 
bearings to tune the system to a desired natural frequency and damping 
factor---this could result in quite a large footer.

A drawback of this concept is difficulty in containing the bearings, the user 
would be free to remove the top piece, and the bearings would all come out of 
their races. Competitors could easily see how the mechanism works and replicate 
it for themselves.

Figure~\vref{fig:concept6} details a variation of C5, whereby the bearings 
travel outwards up the slope. This creates more space for a damping mechanism 
for each race around the outside of the footer. A fluid could be contained in 
the cavity in the bottom piece to damp vibrations using the mechanism described 
in Figure~\ref{fig:fluidic}.

\begin{figure}[h]
    \centering
    \textbf{RUSSELL'S POINTY TOP CONCEPT}
    \caption{C6---Alternative cone and bearing concept.}
    \label{fig:concept6}
\end{figure}
    
The flexibility of the alternative cone and bearing concept C6, coupled with 
its innovative design set the concept apart from its alternatives. In 
\cite{maguire2016vibration}, it was shown how concept C6 with damping mechanism 
D2 best matched the customer's expectations and engineering requirements. This 
is the concept which informed the next stage of development and our final 
design.
    
\section{Detailed Design}

Table~\vref{table:bom} is a bill of materials for the final design. The brand 
name used is \emph{Isofonics}. The items listed are labelled in 
Figure~\vref{fig:labelled-iso-section} and the appendices contain detailed 
drawings for Isofonics parts not purchased from third parties.

\begin{table}[h]
    \centering
    \footnotesize
    \begin{threeparttable}
        \caption{Bill of Materials}
        \label{table:bom}
        \begin{tabular}{@{}clp{17em}c@{}}
            \toprule
            Item No. & Part No. & Description & Qty. \\
            \midrule
            1 & COR080-0003 & \raggedright Isofonics core piece & 1 \\
            2 & TOP080-0004 & \raggedright Isofonics top piece & 1 \\
            3 & 10MMTUNGSTENBALLS\tnote{1} & \raggedright \diameter 10~mm 
            tungsten carbide ball & 6 \\
            4 & F669-N45SH-10\tnote{2} & \raggedright \diameter 10~x~1.5~mm 
            neodymium 
            button magnet & 12 \\
            5 & PLD010-1004 & \raggedright Isofonics preloading back-stop & 1 \\
            6 & RET080-0003 & \raggedright Isofonics retainer & 1 \\
            7 & M4X20-CSK-ST\tnote{3} & \raggedright M4~x~20~mm T20 A2 c'sunk 
            screw with partial thread & 6 \\
            8 & PLD080-0003 & \raggedright Isofonics preloading crown & 1 \\
            9 & M4X20-CSK-H\tnote{4} & \raggedright M4~x~20~mm H2.5 A2 c'sunk 
            screw with partial thread & 1 \\
            10 & RET080-1002 & \raggedright Isofonics retainer bottom & 1 \\
            11 & USR080-0001 & \raggedright Isofonics removable spike & 1 \\
            12 & USR080-1002 & \raggedright Isofonics removable base & 1 \\
            \bottomrule
        \end{tabular}
        \begin{tablenotes}
            \raggedright
            \tiny
            \item[1]\url{http://www.vxb.com/10mm-Tungsten-Carbide-Bearing-Ball-0-3937-inch-Dia-p/10mmtungstenballs.htm}
            \item[2]\url{http://www.first4magnets.com/circular-disc-rod-magnets-c34/10mm-dia-x-1-5mm-thick-n45sh-neodymium-magnet-1-1kg-pull-p3633}
            \item[3]\url{http://www.westfieldfasteners.co.uk/A2_ScrewBolt_PinTXCsk_M4.html}
            \item[4]\url{http://www.westfieldfasteners.co.uk/A2_ScrewBolt_SHCsk_M4.html}
        \end{tablenotes}
    \end{threeparttable}
\end{table}

\begin{figure}[h]
    \centering
    \textbf{LABELLED ISOMETRIC SECTION VIEW}
    \caption{Isometric section view with labelled item numbers.}
    \label{fig:labelled-iso-section}
\end{figure}

The final design for the Isofonics footer can be split into four subsystems, 
each of which will be described in detail:

\begin{itemize}
    \item The isolation and attenuation mechanism.
    \item The preloading mechanism which is used to tweak the system to the 
    supported mass. 
    \item The retaining wall designed to contain the mechanical subsystems and 
    dissuade from disassembly.
    \item User replaceable parts for mounting the system.
\end{itemize}

\subsection{Isolation and Attenuation Mechanism}

Items~1--4 are used in the isolation and attenuation subsystem of the Isofonics 
footer. Of these, Item~3 and Item~4 can be purchased from third parties.

Item~3 is a \diameter 10~mm bearing made of tungsten carbide and sourced from 
VXB Bearings. Despite its high unit price relative to other 
materials---\textsterling25 each---tungsten carbide could be justified for its 
incredible hardness. The hard material avoids resonances of large amplitude and 
does not require lubrication. Any particulates which enter the system are 
ground down by the tungsten carbide.

The neodymium magnets labelled Item~4 were sourced from first4magnets, 
chosen to have the same \diameter 10~mm diameter as the bearings. The 
parametric model described in section~\vref{sec:parametric-model} was used to 
determine whether the 1.5~mm thickness and 1.1~kg steel pull of the part was 
sufficientto support the mass of HiFi amplifiers which range 5--30~kg and 
deliver a natural frequency of 8--12~Hz. Figure~\vref{fig:final-envelope} 
suggests this magnet meets those requirements, as the operational area enclosed 
by the four boundary conditions ranges from 3?--45?~kg.

Item 1 and Item 2 would be machined from 304 stainless steel. This alloy was 
chosen for its hardness, protecting the parts from wear due to moving parts and 
efficiently transmitting vibrations from the equipment to the bearing. 

In order to ensure a point contact between the bearing and the 
are to 

\begin{figure}[h]
    \centering
    \textbf{OPERATIONAL AREA ENVELOPE}
    \caption{Isometric section view with labelled item numbers.}
    \label{fig:final-envelope}
\end{figure}

\printbibliography
    
\end{document}